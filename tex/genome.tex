%\documentclass[conference,compsoc]{IEEEtran}
\documentclass{article}
\usepackage{url,epsfig,ifthen}
\title{What does it mean to own our genes?\\
{\small Genetic De-anonymization Attacks and Implications for Privacy}}
%Draft}}
\author{Arvind Narayanan \\
{\small Stanford University}}

\newcommand{\XX}{{\cal X}}
\newcommand{\YY}{{\cal Y}}
\newcommand{\Adv}{\textsf{Adv}}
\newcommand{\PP}{\textsf {PP}}
\newcommand{\PR}{\textsf {PR}}
\newcommand{\bkg}{\textsf {BkGnd}}
\newcommand{\priv}{\textsf {priv}}
\newcommand{\pub}{\textsf {pub}}
\newcommand{\Aux}{\textsf {Aux}}
\newtheorem{definition}{Definition}

\newcommand{\grumbler}[2]{\begin{quote}\sl{\bf #1:} #2\end{quote}}
\newcommand{\arvind}[1]{\grumbler{Arvind}{#1}}

\newcommand{\ie}{\textit{i.e.}}
\newcommand{\eg}{\textit{e.g.}}

\begin{document}
\maketitle




\begin{abstract}
The right of a person to control his or her own data is a fundamental tenet of privacy advocacy today. This position implicitly assumes the {\em possibility} of such control, at least with the co-operation of the powers that be -- broadly, the government, and those involved in data collection and analysis.

In the context of genetic data, it is well known that each person shares genetic material with blood relatives, and that the extent of sharing is dependent upon the degree of the relationship. Given this fact, how does the notion of an ``owner'' of each piece of data hold up? Does it merely need to be modified, or jettisoned entirely, thus necessitating an overhaul of the genetic privacy debate?

We show that the answer is likely to be disturbingly close to the latter possibility. The key to our conclusion is an analysis of what an adversary, equipped with realistic ``auxiliary information,'' might be able to deduce from completely anonymous genetic material -- for instance, pieces of hair collected {\em en masse} from public spaces without the consent, or even the {\em knowledge}, of the potential victims. We qualify and quantify the conditions under which identities can be affixed to these surreptitiously collected samples, and argue that these conditions will become a reality within a few short years.

There does not appear to be a purely technological fix to our attack. We briefly present policy prescriptions that may delay the eventuality of genetic re-identifiability, and argue that genetic privacy norms must change to accomodate the new technological reality.
\end{abstract}

\section{Introduction}

Threats to genetic privacy have been of great concern to society every since the dawn of DNA fingerprinting. However, the availability of cheap genotyping in the last few years has changed the scale of the threat so much as to make it qualitatively different. For instance, at the time of this writing, a ``personal genomics'' kit costs \$399 \cite{23-and-me} and has been estimated to capture over 50\% of the entropy in the human genome \cite{snpentropy}.

Privacy violations could take many forms, such as data breaches and non-consensual information sharing. Protection against such violations falls under the domain of public policy, law enforcement and computer security. We are instead concerned about {\em inference attacks}, which result in unexpected revelation of sensitive facts about individuals from data or samples that might at first glance appear to hide the facts in question.

Inference attacks have recently received considerable attention: in 2007, Homer {\em et al} showed how to determine  if an individual contributed DNA to a mixture, by analyzing only the {\em aggregate} values of each genotype locus. Further papers have discussed this attack, arguing that it is either less serious \cite{sriram} or more serious \cite{css} than previously thought.  In other work, Goodrich analyzed threats to the privacy of mitochondrial DNA sequences \cite{goodrich}.

Common to these attacks is the fact that the victim needs to have first voluntarily provided an identifiable DNA sample or their genotype information. On the other hand, we consider the question of what an adversary can learn from a completely anonymous genetic sample -- for instance, pieces of hair collected {\em en masse} from public spaces without the consent, or even the {\em knowledge}, of the potential victims.

 Our analysis rests on the assumption that two types of data will become available to powerful adversaries within at most a decade or two: 
\begin{itemize}
\item
a {\em genealogical graph} representing of a significant fraction of all blood relationships among a large population group
\item
{\em genotypes} of a small fraction of the individuals represented in the above graph
\end{itemize}

Fast algorithms to identify shared (``identical-by-descent'') genetic sequences between related individuals have recently been developed \ref{germline}. This enables the adversary, with some confidence, to (probabilistically) detect even remote blood relationships among individuals.

Our main technical contribution is an inference attack which searches for familial relationships between the genotype of an unknown individual and each of a set of identified genotypes, and uses the relationships thus obtained to identify the node in the graph corresponding to the unknown genotype. The algorithm has these properties:
\begin{itemize}
\item
requires only a small fraction of labeled nodes
\item
can tolerate noise
\item
can tolerate missing data
\item
works on a variety of mating models
\item
If the unknown genotype does not correspond to any node in the graph, the  algorithm is able to detect this.
\end{itemize}

Let us walk through a simple example to illustrate how such re-identification could be carried out. Let us call our victim Victor, the individual whose genetic information the adversary possesses.  Victor's node happens to be somewhere in the genealogical graph available to the adversary; the adversary would like to find out if that is in fact the case, and if so, determine the identity of the node. Suppose that two of the victim, Victor's cousins Peggy and Sue have labeled genotypes available in the graph. Furthermore, assume that Victor is related to Peggy through his mother and to Sue through his father.  

[illustration here]

Given this setup, the adversary can now determine that unlabeled genotype comes from someone who shares 1/8 of their DNA with both Peggy and Sue.\footnote{The attacker can deduce more than this: for instance, analyzing the {\em number} of shared segments in addition to their {\em total length} can tell him that the related individuals share a pair of grandparents, rather than one being a great-grandparent of the other, which would also result in a $\frac{1}{8}$ shared genome. However, we ignore this for the purpose of this  example.} Intuitively, if all mating pairs in the family tree are a) monogamous and b) have no recent common ancestor, then the only nodes that satisfy the constraints available to the adversary are Victor, and any siblings he might have. The adversary looks at the graph and determines that the sample came either from Victor, his sister Wanda, or any other siblings he might have who are not represented in the graph.

Since it is impossible to distinguish between siblings without further auxiliary information, we consider the adversary to be successful in this case. In Section \ref{}, we outline how to carry out further de-anonymization using other auxiliary information. Thus, even if Victor has not been completely de-anonymized, it can serve as useful starting point for manual analysis which can complete the de-anonymization. Therefore, our algorithm outputs a sequence of ``sibling groups'' in decreasing order of match probability.

Unlike our toy example above, in reality there are several factors that make the inference far more complex: 

\begin{itemize}
\item
Due to the small fraction of labeled genotypes available to the attacker, the probability that a random node in the graph has a labeled cousin is very small. In fact, our algorithm is successful with so few labeled genotypes that an average node has no fourth cousin or closer that is labeled.
\item
We do not make any assumptions about monogamy or common ancestry among mating pairs. (In Section \ref{}, we provide a detailed analysis for a range of different mating models.)
\item
The graph could be incomplete or have errors. Because of this, you never know the exact relationship, but only approximations.
\end{itemize}

The rest of the paper is organized as follows: in Section \ref{reidentification}, we present a survey of re-identification techniques in computer science and their implications for data privacy. ...




\section{Data collection survey}

Recall that there are two types of data that are crucial to the success of our attack: genealogical data and  genotypes of known individuals. In this section, we present a survey of the relevant privacy laws and business incentives for the collection of this data and blah blah.


\subsection{Genealogical data}
In the light of the different level of availability for records pertaining to living and dead people, not only from official bureaus but also from private organizations such as the LDS church (see below), we divide our survey into these two categories.

Data on deceased persons may not always exist, but if it does exist, it is typically much easier to obtain. On the other hand, data on living persons always exists, to the extent that they know who their parents are, but it may not be easy to obtain because of privacy concerns.

We know survey a) the types of original data sources that can be used to compile genealogical graphs \footnote{it must be borne in mind that reconstructing genealogy will be much easier when multiple sources are combined} b) the existing organizations and databases of genealogical information.

Very broadly, 

\subsubsection{Vital records}

Birth, death, marriage  and divorce records are collectively known as ``vital records.'' In the United States, vital records are typically maintained at the state level. These records are for the most part public, although not always available {\em en masse} in digital form. A summary of availability can be found in \cite{messing}. Some salient points to note are:
\begin{itemize}
\item
Birth records are ``aged'' by between 50 to 75 years, i.e., some information is redacted in records that are less than that age.
\item
Each county maintains records in microfiche form, but most of the larger states aggregate this information in digital form at the state level. 
\item
The authors note that ``as time progresses, public information stored at even the smallest county offices will invariably become digitally available."
\end{itemize}

Due to ``aging'' of birth records, constructing genealogies from vital records is likely to be more feasible for deceased persons than for living persons.

In the United States, there is a drop-off in the availability of vital records before the 1900s. The situation in European countries is much better: In the United Kingdom, for instance, vital records have been collected since at least 1837 \cite{ukbmd.org.uk} and contains an estimated 170 million unique records. Data from prior dates is often available from parish registers from individual counties. \footnote{Need survey of more European countries.} 


Constructing a genealogy from such records is not trivial, but can in large part be automated, since the information recorded (first and last name, place, and approximate date/age) is usually sufficient to identify a person uniquely, even in the absence of unique identifiers like SSNs.  The major missing link is the mother's maiden name, which is sometimes but not always available in birth records.  Conveniently, the authors of \cite{messing} provide a number of heuristics that can be used to infer the mother's maiden name.


The construction of a genealogical tree from vital records is a complex process involving many steps, some of which can be automated to a greater degree than others:
\begin{itemize}
\item {\bf Indexing} of digital data involves converting physical vital records into an indexed digital database of records. Sometimes official vital records bureaus perform this task; other times, it is done by private organizations in two steps:
\begin{itemize}
\item {\bf Digitization} First, the paper records are converted into microfilm or microfiche. These are document storage formats that physically compress the size by a factor of 25 or more.\cite{wikipedia}  The microform documents are then converted to digital images using a microfilm/microfiche reader. 
\item {\bf Transcription} Next, the resulting images are converted into text. This process is  typically ``crowdsourced'' to volunteers participating over the Internet.
\end{itemize}
\item {\bf Analysis} After digitization, the data resides in multiple silos, with no inter-links. Further, only immediate family links (such as a parent-child relationship) can be inferred directly from the records. There are two types of algorithms that are used to analyze this data.
\begin{itemize}
\item {\bf Entity resolution}, also known as identity resolution, and in the genealogy context as record matching or record linking. It involves resolving records across different databases as the same entity, based on a fuzzy matching of attributes such as comparing names via Soundex and more sophisticated algorithms \cite{Lait_and_Randall_An_Assessment_of_Name_Matching_Algorithms}.
\item {\bf Graph merging} operates at a more global level, and tries to merge two genealogical graphs \cite{wilson-merge}. These methods are still in their infancy, and little information exists in the published literature. \cite{anabaptist} describe merging two genealogies of the North American Anabaptist population consisting of around 56,000 and 31,000 individuals respectively. Interestingly, new relationships are created in the merged graph which exist in neither original graph.
%Merging is employed by the LDS in creating the ancestry file.  
\end{itemize}

\end{itemize}

{\bf Ancestry.com}

{\bf FreeBMD} is a UK-based charitable organization that has the goal of creating a free transcription of vital records (called ``BMD," for ``Births, Marriages and Deaths'' in the UK context), and making them freely available on the Internet.  Currently, over 170 million unique records have been transcribed.

{\bf FamilySearch} is a project of the U.S.-based {\bf Church of the  Latter Day Saints.} The LDS church maintains the International Genealogical Index, which has data on 250 million names \cite{familysearch}. There are currently 200 cameras converting vital records into microfilm, and the transcription process is crowdsourced. 

{\bf Geni} 

\subsubsection{``Genealogy 2.0"}

Perhaps the best source of genealogical data is the trend of ``genealogy 2.0'' websites that combine genealogy with social networking. Most of these sites have the explicit goal of colleting a genealogical graph of the entire human population. The typical process is for members to create an account, which also corresponds to their node in the genealogical graph, they then extend their family tree starting from their node. 

Graph merging is  an essential feature of all major genealogical websites including Geni\cite{geni-merge}, onegreatfamily.com\cite{onegreatfamily-merge} and MyHeritage\cite{myheritage-merge}.  It is essential in order to unify the family trees contributed by different individuals.  Merges with high similarity scores happen automatically but others require human intervention.


% linking, record matching, entity resolution, identity resolution, name matching, automated genealogy


\section{Background on re-identification}
The study of data re-identification is a subfield of information science that has mushroomed in the last decade. In 1998, Sweeney \cite{sweeney} showed that Massachusetts hospital records blah blah. 

 Sweeney, Malin and Sweeney, AOL, Netflix, Social networks, Golle, Ohm, case law.

\section{Genetics background and model}
The human genome consists of 23 pairs of chromosomes. 

Autosomes/sex chromosomes

{\bf Meiosis.}

\subsection{Germline}

\section{Attack}

\section{Results}

\section{Discussion}
The genetic information in a person's cells was not created by him or her, nor by any other individual. In this important respect, genetic information differs from most other types of information of commercial value that people ``own,'' such as the books they write or even their shopping history. Our genetic information has evolved since the dawn of our species, and indeed since the dawn of life on earth. Needless to say, the notion of ownership of genetic data has been fraught with philosophical difficulties.

One issue that courts have wrestled with is the right to patent a beneficial allele or cell line. Different U.S. states have different rules, some giving the right to the individual, and others demurring \cite{john-moore-who-owns-your-genetic-information}.

A related issue is  privacy: to what extent does an individual have the right to force a health provider to keep a piece of genetic information (such as a hereditary) secret, given that 1) his blood relatives may well share the same genes 2) it might be in the interest of his relatives to have this information revealed to them, perhaps a matter of life and death?


\section{Glossary of terms}
\bibliographystyle{plain}
\bibliography{all}


\end{document}
